\documentclass[11pt,a4paper]{moderncv}

% moderncv themes
%\moderncvtheme[blue]{casual}                 % optional argument are 'blue' (default), 'orange', 'red', 'green', 'grey' and 'roman' (for roman fonts, instead of sans serif fonts)
\moderncvtheme[blue]{classic}                % idem

% character encoding
\usepackage[utf8]{inputenc}                   % replace by the encoding you are using
\usepackage{color}
\usepackage{hyperref}
\hypersetup{colorlinks,urlcolor=blue,unicode=True}

% adjust the page margins
\usepackage[scale=0.8]{geometry}
%\setlength{\hintscolumnwidth}{3cm}						% if you want to change the width of the column with the dates
%\AtBeginDocument{\setlength{\maketitlenamewidth}{6cm}}  % only for the classic theme, if you want to change the width of your name placeholder (to leave more space for your address details
%\AtBeginDocument{\recomputelengths}                     % required when changes are made to page layout lengths

\definecolor{title-color}{HTML}{00599A}
\definecolor{grey}{HTML}{999999}

% personal data
\firstname{\textcolor{title-color}{Roman}}
\familyname{\textcolor{title-color}{Prokofyev}}
%\title{Resumé title (optional)}               % optional, remove the line if not wanted
\address{}{1700 Fribourg, Switzerland}    % optional, remove the line if not wanted
\phone{+41 26 552 0004}

\mobile{+41 76 529 7448}                    % optional, remove the line if not wanted
%\phone{+49 (0) 30 830 38 124}                      % optional, remove the line if not wanted
\email{roman@exascale.info}                      % optional, remove the line if not wanted
\photo[80pt]{roman}                         % '64pt' is the height the picture must be resized to and 'picture' is the name of the picture file; optional, remove the line if not wanted

% to show numerical labels in the bibliography; only useful if you make citations in your resume
\makeatletter
\renewcommand*{\bibliographyitemlabel}{\@biblabel{\arabic{enumiv}}}
\makeatother

\newcommand*{\cventrynt}[6]{%
    {\bfseries#2}%
    \ifthenelse{\equal{#3}{}}{}{, {\slshape#3}}%
    \ifthenelse{\equal{#4}{}}{}{ at #4}%
    \ifthenelse{\equal{#5}{}}{}{, #5}%
    .\strut%
    \\#1%
    \ifx&#6&%
      \else{\newline{}\begin{minipage}[t]{\linewidth}\setlength{\leftskip}{1.5em}\small#6\end{minipage}\setlength{\leftskip}{0pt}}\fi
    \par\addvspace{1.5em}
    }

% bibliography with mutiple entries
%\usepackage{multibib}
%\newcites{book,misc}{{Books},{Others}}

%\nopagenumbers{}                             % uncomment to suppress automatic page numbering for CVs longer than one page
%----------------------------------------------------------------------------------
%            content
%----------------------------------------------------------------------------------
\begin{document}
\maketitle

{\large\textit{Currently I am a PhD student in CS working in areas of natural language processing and semantic web.  I particularly enjoy open-ended and research-oriented problems and favor tools and technologies that allow me to tackle these problems most effectively, such as Python programming language.
\\\\
My interests also include textual data analytics and big data processing.}}
\vspace{0.5cm}

\section{Technical Skills}
%\subsection{PhD thesis (summa cum laude)}
\cvline{\textbf{Programming Languages}}{\hspace{0.3cm}\emph{Python, Java}}
\cvline{\textbf{Technologies}}{\hspace{0.3cm}\emph{scikit-learn, nltk, hadoop, django, jquery, postgres}}

%\subsection{Master's thesis}
%\cvline{title}{\emph{Navigational SQL, Extending SQL with Entity Navigation}}
%\cvline{supervisors}{Jens Krueger, Prof. Hasso Plattner}
%\cvline{description}{\small Short thesis abstract}
\section{Education}
\cventry{Nov. 2011 -- Present}{PhD Student}{}{University of Fribourg}{}{
Currently I'm a PhD student at the \href{http://exascale.info/}{eXascale Infolab}, under supervision of Professor \href{http://exascale.info/phil}{Philippe Cudré-Mauroux}.
\\
My research is focused on developing new methods that provide effective knowledge extraction from unstructured technical data, knowledge processing and discovery. In particular, I have worked on the following research problems:
\begin{itemize}
\item Tag recommendation for scientific papers based on the the ontology of scientific concepts and user's interest profile.
\item Ontology-based word sense disambiguation for scientific literature.
\item Named Entity Recognition for scientific literature.
\end{itemize}
}%
\cventry{2007--2009}{M.Sc. Applied Mathematics and Physics}{Moscow Institute of Physics and Technology}{Moscow, Russia}{}{
I did my Master's thesis on building enterprise network monitoring and management system based on open source software (\href{http://www.opennms.org}{OpenNMS}).
\\\\
Created initial version of JMX SSL agent for OpenNMS.\\
\href{http://www.opennms.org/wiki/Monitor_several_SNMP_Agents\#Acknowledgement}{http://www.opennms.org/wiki/Monitor\_several\_SNMP\_Agents}
\\\\
GPA: 5 out of 5.
}%
\cventry{2003--2007}{B.Sc. Applied Mathematics and Physics}{Moscow Institute of Physics and Technology}{Moscow, Russia}{}{
During my second year at the university, we wrote a 3D billiards simulator.
The capture of the gameplay is available on youtube:
\\
\href{http://www.youtube.com/watch?v=8Lzr0kWM440}{http://www.youtube.com/watch?v=8Lzr0kWM440}
\\\\
My GPA was 4.7 out of 5.
} 
% arguments 3 to 6 can be left empty
%\cventry{year--year}{Degree}{Institution}{City}{\textit{Grade}}{Description}

\section{Experience}
\cventry{Feb. 2010 -- Mar. 2011}{Python/Django Developer}{}{École Polytechnique Fédérale de Lausanne (EPFL)}{}{
My general responsibilities were to maintain and develop new features for a scientific application running on the Python/Django platform (\href{http://sciencewise.info}{ScienceWISE.info}). The system is processing large amounts of scientific articles from the arxiv.org website on a daily basis. Every article is then linked with scientific concepts appearing in it using the manually-curated ScienceWISE Ontology. The links between articles and concepts are then used to provide a semantic search over the collections of articles.
\\\\
I have also implemented a number of sub-projects that expanded the functionality of the system, in particular:
\begin{itemize}
\item A full text search solution based on the Whoosh engine.
\item A multi-purpose tool to extract taxonomies of scientific concepts from external resources, such as Wikipedia and other online encyclopaedias; and further integrate them to the ScienceWISE Ontology.
\item An authentication system based on SWITCH AAI.
\item A live LaTeX-editor for writing definitions of scientific concepts.
\end{itemize}
\vspace{0.3cm}
The paper presenting the project received the \href{http://iswc2011.semanticweb.org/program/awards/}{best demo paper award} at International Semantic Web Conference 2011.
}%
\cventry{Dec. 2007 -- Jan. 2010}{Configuration Manager}{}{COS\&HT}{}{
During my work at COS\&HT, my primary responsibilities included implementing various infrastructure projects for customers as well as maintenance of company's internal infrastructure solutions. The stack of solutions was primarily based on the enterprise products of Sun Microsystems, such as Sun Java Identity Manager, Sun Secure Global Desktop, etc.
\\
Noticeable customer projects:
\begin{itemize}
\item Lead engineer in implementing Identity Management solution for Russian Federal Customs Service based on Sun Java Identity Manager.
\item Engineer in implementing infrastructure solutions for Russian regional administrations providing governmental and municipal services.
\end{itemize}
}%
\cventry{Jun. 2006 -- Nov. 2007}{Software Engineer}{}{NetCracker}{System Performance dept}{
Our department was dedicated to investigating and solving complex issues with unusually slow performance and memory leaks in J2EE applications. This work required performing lots of complex debugging and analysis of Java heapdumps.
\\\\
I have also implemented a \emph{Jython} application for automated deployment of J2EE applications on \emph{WebLogic Application Server}.
}%

\section{Additional Projects}
\cventry{May 2012 -- Present}{FarPlano}{}{}{}{
Android app for public transportation in Switzerland.\\
\href{https://play.google.com/store/apps/details?id=com.schedulr}{play.google.com/store/apps/details?id=com.schedulr}
}%
\cventry{Sep. 2011 -- Present}{django-selenium}{\href{https://pypi.python.org/pypi/django-selenium/}{pypi.python.org/pypi/django-selenium}}{}{}{
Selenium testing library integration for Django Framework.
}%

%\section{Patents}
%\cvlistitem{Lifecycle Based Horizontal Partitioning (in application) - USP 221624, USP 333803}
%\cvlistitem{Computer-implemented method for operating a database and corresponding computer system (in application) - USP 221624}
%\cvlistitem{Relational Database for Storing Business Objects and Method for Operating the Same (in application) - USP 61307018}            % optional other symbol

%\renewcommand{\listitemsymbol}{-} % change the symbol for lists


% Publications from a BibTeX file without multibib\renewcommand*{\bibliographyitemlabel}{\@biblabel{\arabic{enumiv}}}% for BibTeX numerical labels
\pagebreak
\href{http://scholar.google.com/citations?user=GhGV4P8AAAAJ&hl=en}{Google Scholar Profile}
\nocite{*}
\bibliographystyle{plain}
{\small
\bibliography{publications}       % 'publications' is the name of a BibTeX file
}

% Publications from a BibTeX file using the multibib package
%\section{Publications}
%\nocitebook{book1,book2}
%\bibliographystylebook{plain}
%\bibliographybook{publications}   % 'publications' is the name of a BibTeX file
%\nocitemisc{misc1,misc2,misc3}
%\bibliographystylemisc{plain}
%\bibliographymisc{publications}   % 'publications' is the name of a BibTeX file

\end{document}


%% end of file `template_en.tex'.
