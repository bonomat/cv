\documentclass[11pt,a4paper]{moderncv}

% moderncv themes
%\moderncvtheme[blue]{casual}                 % optional argument are 'blue' (default), 'orange', 'red', 'green', 'grey' and 'roman' (for roman fonts, instead of sans serif fonts)
\moderncvtheme[blue]{classic}                % idem

% character encoding
\usepackage[utf8]{inputenc}                   % replace by the encoding you are using
\usepackage{color}
\usepackage{hyperref}
\hypersetup{colorlinks,urlcolor=blue}

% adjust the page margins
\usepackage[scale=0.8]{geometry}
%\setlength{\hintscolumnwidth}{3cm}						% if you want to change the width of the column with the dates
%\AtBeginDocument{\setlength{\maketitlenamewidth}{6cm}}  % only for the classic theme, if you want to change the width of your name placeholder (to leave more space for your address details
%\AtBeginDocument{\recomputelengths}                     % required when changes are made to page layout lengths

\definecolor{title-color}{HTML}{00599A}
\definecolor{grey}{HTML}{999999}

% personal data
\firstname{\textcolor{title-color}{Roman}}
\familyname{\textcolor{title-color}{Prokofyev}}
%\title{Resumé title (optional)}               % optional, remove the line if not wanted
\address{}{1700 Fribourg, Switzerland}    % optional, remove the line if not wanted
\phone{+41 26 552 0004}

\mobile{+41 76 529 7448}                    % optional, remove the line if not wanted
%\phone{+49 (0) 30 830 38 124}                      % optional, remove the line if not wanted
\email{roman.prokofyev@gmail.com}                      % optional, remove the line if not wanted
\photo[80pt]{roman}                         % '64pt' is the height the picture must be resized to and 'picture' is the name of the picture file; optional, remove the line if not wanted

% to show numerical labels in the bibliography; only useful if you make citations in your resume
\makeatletter
\renewcommand*{\bibliographyitemlabel}{\@biblabel{\arabic{enumiv}}}
\makeatother

\newcommand*{\cventrynt}[6]{%
    {\bfseries#2}%
    \ifthenelse{\equal{#3}{}}{}{, {\slshape#3}}%
    \ifthenelse{\equal{#4}{}}{}{ at #4}%
    \ifthenelse{\equal{#5}{}}{}{, #5}%
    .\strut%
    \\#1%
    \ifx&#6&%
      \else{\newline{}\begin{minipage}[t]{\linewidth}\setlength{\leftskip}{1.5em}\small#6\end{minipage}\setlength{\leftskip}{0pt}}\fi
    \par\addvspace{1.5em}
    }

% bibliography with mutiple entries
%\usepackage{multibib}
%\newcites{book,misc}{{Books},{Others}}

%\nopagenumbers{}                             % uncomment to suppress automatic page numbering for CVs longer than one page
%----------------------------------------------------------------------------------
%            content
%----------------------------------------------------------------------------------
\begin{document}
\textcolor{grey}{Full CV available at: \href{http://careers.stackoverflow.com/prokofyev}{careers.stackoverflow.com/prokofyev}}\\
\maketitle

{\large\textit{I love hacking on ideas and bringing them to working prototypes as quickly as possible, thus in my work I try to favor technologies that allow me to build prototypes in the quickest way. In my opinion, this allows to quickly test many ideas and select those ones that deserve a solid implementation.}}
\vspace{0.5cm}

\section{Technical Skills}
%\subsection{PhD thesis (summa cum laude)}
\cvline{\textbf{Technologies}}{\emph{python, java, django, solr, jquery, mysql, postgres, nginx, apache, memcached, git, mercurial, jenkins, machine-learning, compass-css, css, html5}}

%\subsection{Master's thesis}
%\cvline{title}{\emph{Navigational SQL, Extending SQL with Entity Navigation}}
%\cvline{supervisors}{Jens Krueger, Prof. Hasso Plattner}
%\cvline{description}{\small Short thesis abstract}

\section{Experience}
\cventry{Nov. 2011 -- Present}{Doctoral Assistant}{}{University of Fribourg}{}{
\textbf{Technologies: }{\emph{word sense disambiguation, machine learning, natural language processing, ontology learning, semantic web, nltk, scikit-learn, python, django}}
\\\\
I'm a PhD student at \href{http://exascale.info/}{eXascale Infolab} under supervision of Professor \href{http://exascale.info/phil}{Philippe Cudré-Mauroux}.
My research is concentrated on bringing together existing and developing new methods to provide effective knowledge extraction from unstructured technical data, knowledge processing and discovery. Specifically, the focus is on performing named-entity recognition and disambiguation in scientific documents, establishing semantic relationships between those entities and providing effective document discovery based on the extracted entities.
\\\\
More information and the list of my publications is available at:\\\href{http://exascale.info/members/roman_prokofyev}{http://exascale.info/members/roman\_prokofyev}
}%
\cventry{Feb. 2010 -- Mar. 2011}{Python/Django Developer}{}{École Polytechnique Fédérale de Lausanne}{}{
\textbf{Technologies: }{\emph{python, django, celery, selenium, jquery, git, memcached}}
\\\\
Maintain and develop new features for a web application running on Django/Python (\href{http://sciencewise.info}{sciencewise.info}).
Background task processing system was implemented using \emph{Celery} and \emph{RabbitMQ}.
\\
\begin{itemize}
\item Developed web crawler and HTML parser for collecting data from online information sources
\item Developing rich user interfaces using \emph{JavaScript} and \emph{jQuery}
\item Implemented Continuous Integration system based on \href{http://jenkins-ci.org}{Jenkins CI}
\item Deployed issue tracking and wiki documentation system based on \href{http://www.redmine.org/}{Redmine}
\item Implemented \href{http://www.switch.ch/aai/index.html}{SWITCH AAI authentication system}
\item Implemented document storage system with quota and permission support (google-docs like) - \href{http://sciencewise.info/sciencedocs}{ScienceDocs}
\item Implemented live TeX-editor for creating documents using \href{http://codemirror.net/}{CodeMirror} and \href{http://pyjs.org/}{PyJamas}.
\end{itemize}
\vspace{0.4cm}
The paper describing the project received \href{http://iswc2011.semanticweb.org/program/awards/}{best demo paper award} at ISWC 2011.
}%
\cventry{Dec. 2007 -- Jan. 2010}{Configuration Manager}{}{COS\&HT}{}{
\textbf{Technologies: }{\emph{solaris, linux, confluence, jira, svn, opennms, python}}
\\\\
Maintain internal servers and network, implement new services.
\\
\begin{itemize}
\item Implemented and maintained collaboration product infrastructure with single sign-on (project tracking system, enterprise wiki) based on \href{http://atlassian.com/}{Atlassian products}
\item Implemented and maintained enterprise content management solution with \href{http://www.alfresco.com/}{Alfresco ECM} including LDAP/AD integration
\item Participated in a project building Identity Management System for \emph{Federal Customs Service of Russia} based on \emph{Sun Java Identity Manager}.
\end{itemize}
}%
\cventry{Jun. 2006 -- Nov. 2007}{Software Engineer}{}{NetCracker}{System Performance dept}{
\textbf{Technologies: }{\emph{java, weblogic, jython}}
\\\\
Troubleshooting and optimizing java applications performance.
\\\\
Our department was dedicated to investigating and solving complex issues with slow performance and memory leaks in J2EE applications. This work required performing lots of complex debugging and Java heapdump analysis.
\\\\
Designed, developed and implemented a \emph{Jyhton} application for automated Java Enterprise application deployment on \emph{WebLogic Application Server}.
}%

\section{Education}
\cventry{2007--2009}{M.Sc. Applied Mathematics and Physics}{Moscow Institute of Physics and Technology}{Moscow, Russia}{}{
I did my Master's thesis on building enterprise network monitoring and management system based on the open source software (\href{http://www.opennms.org}{OpenNMS}).
\\\\
Created initial version of JMX SSL agent for OpenNMS.\\
\href{http://www.opennms.org/wiki/Monitor_several_SNMP_Agents\#Acknowledgement}{http://www.opennms.org/wiki/Monitor\_several\_SNMP\_Agents}
}%
\cventry{2003--2007}{B.Sc. Applied Mathematics and Physics}{Moscow Institute of Physics and Technology}{Moscow, Russia}{}{
During my second year at the university, we wrote a 3D billiards simulator.
The capture of the gameplay is available on youtube:
\\
\href{http://www.youtube.com/watch?v=8Lzr0kWM440}{http://www.youtube.com/watch?v=8Lzr0kWM440}
\\\\
My GPA was 4.7 out of 5.
} 
% arguments 3 to 6 can be left empty
%\cventry{year--year}{Degree}{Institution}{City}{\textit{Grade}}{Description}

\section{Projects}
\cventry{May 2012 -- Present}{Farplano (Swiss Stationboards)}{\href{https://play.google.com/store/apps/details?id=com.schedulr}{play.google.com}}{}{}{
\textbf{Technologies: }{\emph{jqtouch, javascript, phonegap, heroku, android, pagodabox, java, mobile}}
\\\\
Public Transport Stationboards mobile app for all public transport in Switzerland and railway transport in Europe.
\\\\
Application was initially built using \emph{phonegap + jqtouch} bundle, then re-implemented natively for Android $\geq$ 4.0.
\\\\
More on the development process, different app version and data analytics:\\ \href{http://prokofyev.ch/categories/stationboards-app.html}{prokofyev.ch/categories/stationboards-app.html}
\\\\
\textbf{Role:} Project Lead and Main Developer.
}%
\cventry{Sep. 2011 -- Present}{django-selenium}{\href{https://github.com/dragoon/django-selenium}{github.com/dragoon/django-selenium}}{}{}{
\textbf{Technologies: }{\emph{python, django, selenium, tox, travis-ci}}
\\\\
Selenium testing library integration for django, syntactic sugar proxy to Selenium WebDriver.
Usage statistics are available on crate.io: \href{https://crate.io/packages/django-selenium/}{crate.io/packages/django-selenium}
\\\\Not actively maintained.
\\\\
\textbf{Role:} Lead Developer.
}%
\cventry{May 2011 -- Jun. 2013}{Prestashop Sync Service}{\href{http://prestashop-sync.com}{prestashop-sync.com}}{}{}{
\textbf{Technologies: }{\emph{django, python, celery, jquery, scss, compass-css, selenium, php}}
\\\\
Prestashop Sync Service is a web service tool that allows shop owners to easily synchronize product prices and quantities in their shops. The project allows to perform unattended synchronization via CSV files.
\\\\
\textbf{Role:} Project Lead and Main Developer.
}%

\section{Languages}
\cvlanguage{Russian}{Native proficiency}{}
\cvlanguage{English}{Professional working proficiency}{}
\cvlanguage{French}{Elementary proficiency}{}

%\section{Patents}
%\cvlistitem{Lifecycle Based Horizontal Partitioning (in application) - USP 221624, USP 333803}
%\cvlistitem{Computer-implemented method for operating a database and corresponding computer system (in application) - USP 221624}
%\cvlistitem{Relational Database for Storing Business Objects and Method for Operating the Same (in application) - USP 61307018}            % optional other symbol

%\renewcommand{\listitemsymbol}{-} % change the symbol for lists


% Publications from a BibTeX file without multibib\renewcommand*{\bibliographyitemlabel}{\@biblabel{\arabic{enumiv}}}% for BibTeX numerical labels
\nocite{*}
\bibliographystyle{plain}
{\small
\bibliography{publications}       % 'publications' is the name of a BibTeX file
}

% Publications from a BibTeX file using the multibib package
%\section{Publications}
%\nocitebook{book1,book2}
%\bibliographystylebook{plain}
%\bibliographybook{publications}   % 'publications' is the name of a BibTeX file
%\nocitemisc{misc1,misc2,misc3}
%\bibliographystylemisc{plain}
%\bibliographymisc{publications}   % 'publications' is the name of a BibTeX file

\end{document}


%% end of file `template_en.tex'.
